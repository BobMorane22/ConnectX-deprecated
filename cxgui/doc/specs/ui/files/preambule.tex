% ------------------------------------------------------------------------------
% Connect X project's GUI specification LaTeX document preambule.
%
% @file     preambule.tex
% @author   Éric Poirier
% @date     December, 2017
% @version  1.0
% ------------------------------------------------------------------------------

% ------------------------------------------------------------------------------
% Copyright (C) 2017 Connect X team.
%
% This file is part of Connect X.
%
% Connect X is free software: you can redistribute it and/or modify
% it under the terms of the GNU General Public License as published by
% the Free Software Foundation, either version 3 of the License, or
% (at your option) any later version.
%
% Connect X is distributed in the hope that it will be useful,
% but WITHOUT ANY WARRANTY; without even the implied warranty of
% MERCHANTABILITY or FITNESS FOR A PARTICULAR PURPOSE.  See the
% GNU General Public License for more details.
%
% You should have received a copy of the GNU General Public License
% along with Connect X.  If not, see <http://www.gnu.org/licenses/>.
% ------------------------------------------------------------------------------

% PACKAGES:

  % General document properties:
  \usepackage[titletoc, title]{appendix}
  \usepackage[colorlinks=true, allcolors=black, plainpages=false,pdfpagelabels]{hyperref}
  \usepackage[nottoc]{tocbibind}
  \usepackage[margin = 1.25in]{geometry}
  \usepackage{setspace} \onehalfspacing
  \usepackage{multicol}


  % Language options:
  \usepackage{lmodern}
  \usepackage[utf8]{inputenc}
  \usepackage[T1]{fontenc}
  \usepackage[main = english, french]{babel}

  % Figures options:
  \usepackage{graphicx}
  \usepackage{wrapfig}
  \usepackage[font = small]{caption}
  \usepackage{subcaption}
  \usepackage{tikz}
  \usepackage{floatrow}
  \usepackage{pdfpages}
  \usepackage{cleveref}


  % Math options:
  \usepackage{amsmath}
    \numberwithin{equation}{section}

  % Source code options:
  \usepackage{xcolor}
  \usepackage{listingsutf8}
  \renewcommand{\lstlistingname}{Source}
  \renewcommand{\lstlistlistingname}{Liste des \lstlistingname s}

    % Source code formatting:
    \definecolor{forestgreen}{rgb}{0.15,0.42,0.16}

    \lstset{
           backgroundcolor = \color{lightgray},
           basicstyle = \footnotesize\ttfamily\color{black}\bfseries,
           breakatwhitespace = false,
           breaklines = true,
           captionpos = b,
           commentstyle = \color{forestgreen},
           deletekeywords = {...},
           escapeinside = {\%*}{*)},
           extendedchars = true,
           frame = single,
           inputencoding = utf8/latin1,
           keepspaces = true,
           keywordstyle = \color{blue},
           otherkeywords = {*,...},
           numbers = left,
           numbersep = 5pt,
           numberstyle = \tiny\color{black},
           rulecolor = \color{black},
           showspaces = false,
           showstringspaces = false,
           showtabs = false,
           stepnumber = 1,
           stringstyle = \color{black},
           tabsize = 4,
           %title = \lstname
           literate={à}{{\`a}}1 {è}{{\`e}}1 {é}{{\'e}}1 {ê}{{\^e}}1 {ù}{{\`u}}1
           }
