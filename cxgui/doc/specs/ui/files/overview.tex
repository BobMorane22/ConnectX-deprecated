The \ttt{Connect X} grahical user interface will consist of four main dialogs,
which will be briefly described here, and of other second order dialogs like
information or warning dialogs, where appropriate. The dialog where the players
will spend most of their time will be the Game dialog:

\begin{figure}[H]
  \includegraphics[width = 0.5\textwidth]{\proto MainWindow.png}
  \caption{A game dialog prototype}
  \label{fig:Game}
\end{figure}

This dialog contains most of the game features: the game board and the discs,
information on the players and the game, player actions, such as placing a
disc and creating a new game. It is also from this dialog that all the other
dialogs can be invoked.

The second most important dialog is the new game dialog:

\begin{figure}[H]
  \includegraphics[width = 0.4\textwidth]{\proto NewGame.png}
  \caption{A new game dialog prototype}
  \label{fig:NewGame}
\end{figure}

This dialog is needed to create a new game. A new game must be created upon
starting up Connect X and if some game parameters must be changed (i.e if
a new player wants to join or the in-a-row value must be changed). Another
helpful dialog is the help dialog (!):

\begin{figure}[H]
  \includegraphics[width = 0.4\textwidth]{\proto Help.png}
  \caption{A help dialog prototype}
  \label{fig:Help}
\end{figure}

All this dialog does is redirect the user to the online help for \ttt{Connect X}.
You can also take a look at the about dialog, which gives some more technical
information on the software, such as the developper credits and the licencing:

\begin{figure}[H]
  \includegraphics[width = 0.3\textwidth]{\proto About.png}
  \caption{An about dialog prototype}
  \label{fig:AboutCX}
\end{figure}
