% ------------------------------------------------------------------------------
% Connect X project's programming standard section on control structure.
%
% Explains the basic guidelines on how to use control structures when 
% needed.
%
% @file     control.tex
% @author   Éric Poirier
% @date     March, 2017
% @version  1.0
% ------------------------------------------------------------------------------

% ------------------------------------------------------------------------------
% Copyright (C) 2017 Connect X team.
%
% This file is part of Connect X.
%
% Connect X is free software: you can redistribute it and/or modify
% it under the terms of the GNU General Public License as published by
% the Free Software Foundation, either version 3 of the License, or
% (at your option) any later version.
%
% Connect X is distributed in the hope that it will be useful,
% but WITHOUT ANY WARRANTY; without even the implied warranty of
% MERCHANTABILITY or FITNESS FOR A PARTICULAR PURPOSE.  See the
% GNU General Public License for more details.
%
% You should have received a copy of the GNU General Public License
% along with Connect X.  If not, see <http://www.gnu.org/licenses/>.
% ------------------------------------------------------------------------------

\section{Prefer iterators}
Iterators in C++ provide a way to generalize a looping process to all 
\ttt{std}-like data containers. No random access operator is needed to perform 
the loop. For example:

\includecode{\controlSnippets iterator.cpp}

In the above example, the \ttt{std::string} could be replaced by an 
\ttt{std::list}, even if no random access operator is defined for it. Iterators 
should always be preffered to simple \ttt{for}-loops. They are welcomed allied 
for efficient refactoring!

\section{Use range \ttt{for} loops}
Whenever possible, prefer the range-\ttt{for}-loop. It is one of the best self 
documenting tool available in C++11. However, be careful with the keyword 
\ttt{auto} not to generate unwanted copies of objects (especially for large 
ones!). Consider the following example:

\includecode{\controlSnippets auto.cpp}

The second example produces the expected result and is cheaper for no copy is 
done. The first loop is just wrong since the \ttt{format()} method is only 
applied to the copy!

\section{Nesting is evil}
Whenever you can, try to avoid loop nesting and even logic nesting. Instead, 
use more functions or \ttt{std::algorithm}. Nested structures are often hard 
to follow, allow more mistakes to be made and are poorly self documented.

\section{Try to avoid loops}
If you can, use an \ttt{std::algorithm} instead, especially if it makes reading 
the code clearer. Remember that algorithms are well tested allies!