% ------------------------------------------------------------------------------
% Connect X project's programming standard section on code documentation.
%
% This sections explains how to use the Doxygen framework to document the 
% code and argue that inline documentation should be reduced to a minimum.
% Code self documentation should be the real goal.
%
% @file     documentation.tex
% @author   Éric Poirier
% @date     March, 2017
% @version  1.0
% ------------------------------------------------------------------------------

% ------------------------------------------------------------------------------
% Copyright (C) 2017 Connect X team.
%
% This file is part of Connect X.
%
% Connect X is free software: you can redistribute it and/or modify
% it under the terms of the GNU General Public License as published by
% the Free Software Foundation, either version 3 of the License, or
% (at your option) any later version.
%
% Connect X is distributed in the hope that it will be useful,
% but WITHOUT ANY WARRANTY; without even the implied warranty of
% MERCHANTABILITY or FITNESS FOR A PARTICULAR PURPOSE.  See the
% GNU General Public License for more details.
%
% You should have received a copy of the GNU General Public License
% along with Connect X.  If not, see <http://www.gnu.org/licenses/>.
% ------------------------------------------------------------------------------

\section{Documentation tools}
Connect X uses Doxygen to document APIs. Some madatory pieces of 
documentation have been designed to ensure complete documentation 
for the developpers.

\section{Public API mandatory documentation}
All files should start with:
\includecode{\documentationSnippets file.cpp}

All methods and functions should start with:
\includecode{\documentationSnippets function.cpp}
Some private methods or functions may not be documented but it is good practice 
to do so. Doxygen will ignore this documentation, but it may help fellow 
developpers in the future.

All classes and structures should start with:
\includecode{\documentationSnippets class.cpp}

All attributes should also be documented as follow:
\includecode{\documentationSnippets attribute.cpp}

Other pieces of code (\ttt{namespace}s, \ttt{typedef}s, etc) should also be 
documented using the Doxygen tags, but a normalized documentation format 
has not been designed.

\section{Internal documentation}
One should only add comment to internal code (for example to explain a line 
inside the body of a function) after careful consideration. Usually, this 
practice is a sign of bad design (but not always) and poor self documenting 
code. The code should speak for itself. Here are some ideas on how to avoid 
comments in internal code:

\begin{enumerate}

 \item separate the code in smaller functions with meaningful names;
 
 \item code what you want to communicate (for example, use \ttt{const} to say 
       that your method does not modify an object, instead of writting a 
       comment);
       
 \item use well know functions wherever possible (for example, in the excellent 
       \ttt{algorithm} STL header.
       
\end{enumerate}

For example avoid this:
\includecode{\documentationSnippets useless.cpp}

\newpage

and write this instead:
\includecode{\documentationSnippets inTheCode.cpp}