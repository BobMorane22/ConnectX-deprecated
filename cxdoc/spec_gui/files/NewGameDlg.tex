Here is the layout for the new game dialog:

\begin{figure}[H]
  \includegraphics[width = 0.4\textwidth]{\proto NewGame.png}
  \caption{The new game dialog layout}
  \label{fig:NewGameDlgLayout}
\end{figure}

The dialog shall contain the following elements:

\begin{enumerate}

    \item \texttt{Dialog}
              \begin{description}
                  \item[Title] New game
                  \item[Close button] Yes
                  \item[Minimize button] Yes
                  \item[Resize button] Yes
                  \item[Modal] Yes
              \end{description}

    \item \texttt{Section}
              \begin{description}
                  \item[Label] Game
                  \item[Readonly] Yes
                  \item[Folding] No
                  \item[Visible] Yes
                  \item[Tooltip] No
              \end{description}

    \item \texttt{Label}
              \begin{description}
                  \item[Label] In-a-row value:
                  \item[Readonly] Yes
                  \item[Visible] Yes
                  \item[Tooltip] No
              \end{description}

    \item \texttt{Edit box} \label{enum:InARowEditBox}
              \begin{description}
                  \item[Default value] 4
                  \item[Value limits] Integer between three and the minimum between
                                      the number of rows for the game board and the
                                      number of columns in the game board.
                  \item[Readonly] No
                  \item[Enabled] Yes
                  \item[Visible] Yes
                  \item[Tooltip] Yes: \textit{The in-a-row value must be at least 3
                                      and fit in the game board for all orientations.}
                  \item[Events] No
                  \item[Misc] The edit box is only validated when the start button is
                              pressed. See \cref{enum:StartButton}.
              \end{description}

    \item \texttt{Section}
              \begin{description}
                  \item[Label] Game board
                  \item[Readonly] Yes
                  \item[Folding] No
                  \item[Visible] Yes
                  \item[Tooltip] No
              \end{description}

    \item \texttt{Label}
              \begin{description}
                  \item[Label] Number of rows:
                  \item[Readonly] Yes
                  \item[Visible] Yes
                  \item[Tooltip] No
              \end{description}

    \item \texttt{Edit box} \label{enum:RowEditBox}
              \begin{description}
                  \item[Default value] 6
                  \item[Value limits] Integer between 6 and 64 included.
                  \item[Readonly] No
                  \item[Enabled] Yes
                  \item[Visible] Yes
                  \item[Tooltip] Yes: \textit{The number of rows must be at least 6
                                              and must not exceed 64.}
                  \item[Events] No
                  \item[Misc] The edit box is only validated when the start button is
                              pressed. See \cref{enum:StartButton}.
              \end{description}

    \item \texttt{Label}
              \begin{description}
                  \item[Label] Number of columns:
                  \item[Readonly] Yes
                  \item[Visible] Yes
                  \item[Tooltip] No
              \end{description}

    \item \texttt{Edit box} \label{enum:ColumnEditBox}
              \begin{description}
                  \item[Default value] 7
                  \item[Value limits] Integer between 7 and 64.
                  \item[Readonly] No
                  \item[Enabled] Yes
                  \item[Visible] Yes
                  \item[Tooltip] Yes: \textit{The number of columns must be at least 7
                                              and must not exceed 64.}
                  \item[Events] No
                  \item[Misc] The edit box is only validated when the start button is
                              pressed. See \cref{enum:StartButton}.
              \end{description}

    \item \texttt{Section}
              \begin{description}
                  \item[Label] Players
                  \item[Readonly] Yes
                  \item[Folding] No
                  \item[Visible] Yes
                  \item[Tooltip] No
              \end{description}

    \item \texttt{Player table} \label{enum:PlayerTable}
              \begin{description}
                  \item[General idea] The player table is a table containing three
                                      columns titles respectively \ttt{No.},
                                      \ttt{Name} and \ttt{Disc}. Each table line
                                      is formed of three cells, each containing
                                      respectively: the row number, the player's
                                      name and the disc color for the player.
                  \begin{description}
                      \item[Row number]  The row number is an integer value from
                                         1 to 10. It shall be readonly and visible.
                      \item[Player name] The player name is a non-empty string value
                                         with initial value "--Player~$n$--", where
                                         $n$ represents the player's row number
                                         value.
                      \item[Disc color] The disc color cell shall be an edit box
                                        showing initial value "Choose color..." and
                                        proposing a choice of 10 different colors.
                  \end{description}
                  \item[Readonly] Yes, but entire lines can be deleted.
                  \item[Default values] Here is how the initial table should look
                                        like:

                                        \begin{table}[H]
                                            \centering
                                            \begin{tabular}{|c|c|c|}
                                                \hline
                                                No. & Name & Disc \\ \hline\hline
                                                1 & --Player 1-- & \fcolorbox{red}{red}{\textcolor{red}{mmmm}}       \\\hline
                                                2 & --Player 2-- & \fcolorbox{black}{black}{\textcolor{black}{mmmm}} \\\hline
                                            \end{tabular}
                                            \caption{Default values for the
                                                     \ttt{Player table}.}
                                            \label{tab:DefaultValuesPlayerTable}
                                        \end{table}

                  \item[Limits] The table shall contain a minimum of 2 lines
                                and a maximum of 10 lines.
                  \item[Visible] Yes
                  \item[Tooltip] No
                  \item[Events] When a player is registered (see \cref{enum:AddPlayer})
                                the table is automatically updated in the following
                                way:
                                \begin{enumerate}
                                    \item A new row is added.
                                    \item The row number cell contains the incremented
                                          number of rows ($n$).
                                    \item The "--Player $n$--" player name is added to the
                                          name cell.
                                    \item If $n = 10$, the add player button is deactivated.
                                          (see \cref{enum:AddPlayer}).
                                \end{enumerate}
                                Players can be removed from the tables until two players
                                are left. Clicking on the row number and hitting \Del
                                shall remove a player. Alternatively, right clicking
                                on the player row and choosing "remove" from the menu
                                shall remove a player from the list. Once a player is
                                removed, the player table shall be updated in the following
                                way:
                                \begin{enumerate}
                                    \item If $n > 2$, the row disapears.
                                    \item If the row is not the end row, all player numbers,
                                          either in their default names or ther No column
                                          are adjusted in such a way that they flow from top
                                          top to bottom of the table from 1 to $n$. There
                                          shall be no jumps.
                                    \item The color that was chosen by the player who's
                                          row is removed shall return to the pool of possible
                                          disc colors to choose from.
                                    \item If $n = 2$, selecting a row and pressing \Del
                                          shall not remove a player.
                                    \item If $n = 2$, the "delete" option from the right
                                          click menu shall be deactivated.
                                \end{enumerate}
                                When a color is chosen by a registered player, the disc
                                color selection edit box shall remove this color from
                                the list of available colors to choose from.
              \end{description}

    \item \texttt{Button} \label{enum:AddPlayer}
              \begin{description}
                  \item[Label] Add player
                  \item[Readonly] Yes
                  \item[Enabled] If there are less than 10 registered players, the
                                 button shall be enabled. Otherwise, it shall be disabled.
                  \item[Visible] Yes
                  \item[Tooltip] Yes: \textit{You can register a maximum of 10 players.}
                  \item[Events] Triggering this button shall add an entry to the player
                                table (see \cref{enum:PlayerTable}).
              \end{description}

    \item \texttt{Button}  \label{enum:StartButton}
              \begin{description}
                  \item[Label] Start
                  \item[Readonly] Yes
                  \item[Enabled] The button shall be enabled when all the following
                                 conditions are met:
                      \begin{itemize}
                          \item There is a value in the in-a-row edit box
                                (see \cref{enum:InARowEditBox}).
                          \item There is a value in the number of rows edit box
                                (see \cref{enum:RowEditBox}).
                          \item There is a value in the number of columns edit box
                                (see \cref{enum:ColumnEditBox}).
                          \item All players have chosen a color
                                (see \cref{enum:PlayerTable}).
                      \end{itemize}
                  \item[Visible] Yes
                  \item[Tooltip] No
                  \item[Alternate control] No
                  \item[Events] The button shall validate all the fields from the dialog
                                when clicked. This validation shall take place before any
                                other action can be triggered. The following elements shall
                                be validated according to their specifications:
                                \cref{enum:InARowEditBox}, \cref{enum:RowEditBox} and
                                \cref{enum:ColumnEditBox}. Furthermore, the following
                                validation shall take place: the players must all have an
                                equal amount of game moves at game start, according to the
                                dialog parameters and the number of registered players.
              \end{description}

\end{enumerate}

%%% Local Variables:
%%% mode: latex
%%% TeX-master: "../connectXGuiSpec"
%%% End: