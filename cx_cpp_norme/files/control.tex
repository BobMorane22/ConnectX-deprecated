\section{Prefer iterators}
Iterators in C++ provide a way to generalize a looping process to all 
\ttt{std}-like data containers. No random access operator is needed to perform 
the loop. For example:

\includecode{\controlSnippets iterator.cpp}

In the above example, the \ttt{std::string} could be replaced by an 
\ttt{std::list}, even if no random access operator is defined for it. Iterators 
should always be preffered to simple \ttt{for}-loops. They are welcomed allied 
for efficient refactoring!

\section{Use range \ttt{for} loops}
Whenever possible, prefer the range-\ttt{for}-loop. It is one of the best self 
documenting tool available in C++11. However, be careful with the keyword 
\ttt{auto} not to generate unwanted copies of objects (especially for large 
ones!). Consider the following example:

\includecode{\controlSnippets auto.cpp}

The second example produces the expected result and is cheaper for no copy is 
done. The first loop is just wrong since the \ttt{format()} method is only 
applied to the copy!

\section{Nesting is evil}
Whenever you can, try to avoid loop nesting and even logic nesting. Instead, 
use more functions or \ttt{std::algorithms}. Nested structures are often hard 
to follow, allow more maistakes to be made and are poorly self documented.

\section{Try to avoid loops}
If you can, use an \ttt{std::algorithm} instead, especially if it makes reading 
the code clearer. Remember that algorithms are well tested allies! See the 
\cref{chapterAlgorithms} for more examples of their use!