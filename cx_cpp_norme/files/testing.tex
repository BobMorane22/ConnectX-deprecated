Testing the code is one of the most important part of software 
developpement. As the codebase grows, tests are added. At some 
point, the tests collection becomes huge and navigating through 
it a real pain, endangering its further use and maintenance. If 
it starts adding too musch overhead to the developper's work, 
there is always the rish that he (the developper) starts neglecting 
it.

\section{Unit testing}

Unit testing aims to test one indivisible entity (a unit). By definition, 
it is easy to know exactly what they do and this knowledge needs 
to be reflected in the name. The rule is that unit test names need to 
always have three parts:

\begin{description}
 \item[Method name]        The name of the method (or function) that is tested.
 
 \item[State under test]   The state of the testes unit, at the moment 
                           it is tested.

 \item[Expected behaviour] The way the unit is expected to react, or behave 
                           when tested in these conditions.
\end{description}

These Three fields should be concatenated using underscores, like so: 

\begin{center}
 \ttt{MethodName\_StateUnderTest\_ExpectedBehaviour}
\end{center}

For examples, test names could look like:

\begin{enumerate}
 \item \ttt{IsAdult\_AgeLessThan18\_ReturnFalse}
 \item \ttt{WithdrawMoney\_InvalidAccount\_ExceptionThrown}
 \item \ttt{AdmitStudent\_MissingMandatoryFields\_ReturnFailToAdmit}
\end{enumerate}

Regarding methods \ttt{isAdult()}, \ttt{withdrawMoney()} and 
\ttt{admitStudent()}. With such a naming convention and a clear 
test body, unit test should not need a lot of documentation. Try to 
keep it minimal for there might be thousands of tests.