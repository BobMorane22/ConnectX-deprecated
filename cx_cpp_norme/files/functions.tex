The following sections apply both to methods and functions.

\section{Parameter passing}
Parameters, when they represent objects (custom or not) should always be passed 
as constant references to avoid uneccessary copying. In other words, a 
parameter named \ttt{p\_aParameter} of type \ttt{Object} should be passed as:

\begin{center}
\ttt{const Object\& p\_aParameter}.
\end{center}

If the parameter is of a fundamental data type and is not an object (\ttt{int}, 
\ttt{bool}, \ttt{double}, etc), this rule does not apply since copy is cheap. 
For example:

\includecode{\functionsSnippets parameters.cpp}

\section{\ttt{const} correctness}
\ttt{const} correctness is perhaps one of the most important rules introduced 
in this document. If a method does not modify the object it acts upon, mark it 
as \ttt{const}. Consider the following example:

\includecode{\functionsSnippets const.cpp}
which shows how \ttt{const} correctness helps avoid tricky mistakes by running 
checks at compile time!

\section{Other specifiers}
The \ttt{delete} keyword should be added to methods which should not exist in a 
class (usually because they don't make sense in the context of the class). For 
example, in the class \ttt{Person} below, the programmer has decided that 
instanciating am object without specifying a name should not be done. 
Therefore, he deleted the default constructor:

\includecode{\functionsSnippets delete.cpp}

The \ttt{override} keyword is also very important. The following example shows 
how it can be used to avoid errors using the compiler:
\includecode{\functionsSnippets override.cpp}

In this case, the \ttt{override} addition makes the compiler complain that the 
method with the signature

\begin{center}
 \ttt{int foo();}
\end{center}
does not exist in the base class. This tells the programmer that the 
\ttt{const} keyword is absent from the signature instead of generating a "new" 
non \ttt{const} \ttt{foo()} method (Notice in the example how the keyword 
\ttt{virtual} is repeated in the overriden method signature. This is good 
practice!).

These two keywords should always be used, when applicable.

Other keywords such as \ttt{default} and \ttt{final} can be used but are not 
mandatory.

\section{Virtual destructors}
Destructor should always be made \ttt{virtual} to ensure correct behaviour if 
the class is eventually derived. This rule does not apply if a class is made 
\ttt{final}, of course.

\section{Inlining}
One line methods should be inlined to ensure correct optimization at compile 
time. If a method has more than one line, do not inline! Example:
\includecode{\functionsSnippets inlining.cpp}

Prefer \ttt{inline} to MACROs, whenever possible. This avoid tricky copy and 
paste errors which may be hard to find:
\includecode{\functionsSnippets inlineVsMacros.cpp}