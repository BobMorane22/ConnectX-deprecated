% ------------------------------------------------------------------------------
% Connect X project's programming standard section on class-like elements.
%
% This section describes some limitations on how classes (and other class-like 
% elements can be used.
%
% @file     varClassStruct.tex
% @author   Éric Poirier
% @date     March, 2017
% @version  1.1
% ------------------------------------------------------------------------------

% ------------------------------------------------------------------------------
% Copyright (C) 2017 Connect X team.
%
% This file is part of Connect X.
%
% Connect X is free software: you can redistribute it and/or modify
% it under the terms of the GNU General Public License as published by
% the Free Software Foundation, either version 3 of the License, or
% (at your option) any later version.
%
% Connect X is distributed in the hope that it will be useful,
% but WITHOUT ANY WARRANTY; without even the implied warranty of
% MERCHANTABILITY or FITNESS FOR A PARTICULAR PURPOSE.  See the
% GNU General Public License for more details.
%
% You should have received a copy of the GNU General Public License
% along with Connect X.  If not, see <http://www.gnu.org/licenses/>.
% ------------------------------------------------------------------------------

\section{\texttt{class} vs \texttt{struct}}
A \texttt{struct} should be defined only when the data members that are 
considered form a data aglomeration without specific behaviour. For example, 
the following data collection is only a way to structure two \texttt{double}s 
but follow no specific behaviour and should be defined inside a \texttt{struct}:

\includecode{\varClassStructSnippets struct.cpp}

In the other hand, the following data agglomeration holds a specific behaviour 
-- a way to calculate its length:

\includecode{\varClassStructSnippets class.cpp}

and therefore should be defined inside a \texttt{class}. Note that 
\texttt{struct}s can sometimes hold constructors and methods. Usually, the 
content of a \texttt{struct} should be completely public and focus should 
be put on the data, not the behaviour.

\section{Access qualifiers}
When thinking about access qualifiers, the principle of least privilege should 
be applied. This principle states that (from Wikipedia):

\begin{quote}
  Every module must be able to access only the information and 
  resources that are necessary for its legitimate purpose.
  \cite{noauthor_principle_2016}
\end{quote}


\section{Inheritance}
Single inheritance is encouraged, multiple inheritance is discouraged other 
than for interface implementation. Consider the following classes and 
interfaces:

\includecode{\varClassStructSnippets baseClass.cpp}

The \texttt{Foo} and \texttt{Waldo} classes are regular classes and should be 
inherited one at a time (even if the class is abstract, but contains some 
implementation--that is, it is not an interface). In the other hand, both 
classes \texttt{IBar} and \texttt{IBaz} are interfaces an can be inherited 
simultaneously. For example:

\includecode{\varClassStructSnippets inheritance.cpp}

Unless you really know what you are doing, all inheritance should be 
\ttt{public}.