\section{\texttt{class} vs \texttt{struct}}
A \texttt{struct} should be defined only when the data members that are 
considered form a data aglomeration without specific behaviour. For example, 
the following data collection is only a way to structure two \texttt{double}s 
but follow no specific behaviour and should be defined inside a \texttt{struct}:

\includecode{\varClassStructSnippets struct.cpp}

In the other hand, the following data agglomeration holds a specific behaviour 
-- a way to calculate its length:

\includecode{\varClassStructSnippets class.cpp}

and therefore should be defined inside a \texttt{class}. Note that 
\texttt{struct}s can sometimes hold constructors and methods. Usually, the 
content of a \texttt{struct} should be completely public.

\section{Access qualifiers}
When thinking about access qualifiers, the principle of least privilege should 
be applied. Also, make sure the public part of your class is as stable as 
possible, especially if it is to be included in a public API!


\section{Inheritance}
Single inheritance is encouraged, multiple inheritance is discouraged other 
than for interface implementation. Consider the following classes and 
interfaces:

\includecode{\varClassStructSnippets baseClass.cpp}

The \texttt{Foo} and \texttt{Waldo} classes are regular classes and should be 
inherited one at a time (even if the class is abstract, but contains some 
implementation--that is, it is not an interface). In the other hand, both 
classes \texttt{IBar} and \texttt{IBaz} are interfaces an can be inherited 
simultaneously. For example:

\includecode{\varClassStructSnippets inheritance.cpp}

Unless you really know what you are doing, all inheritance should be public.