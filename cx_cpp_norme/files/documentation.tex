\section{Documentation tools}
Connect X uses Doxygen to document APIs. Some madatory pieces of 
documentation have been designed to ensure complete documentation for the users 
(and the developpers!).

\section{Public API mandatory documentation}
All files should start with:
\includecode{\documentationSnippets file.cpp}

All methods and functions should start with:
\includecode{\documentationSnippets function.cpp}
Some private methods or functions may not be documented but it is good practice 
to do so. Doxygen will ignore this documentation, but it may help fellow 
developpers in the future.

All classes and structures should start with:
\includecode{\documentationSnippets class.cpp}

All attricutes should alwo be documented as follow:
\includecode{\documentationSnippets attribute.cpp}

Other pieces of code (\ttt{namespace}s, \ttt{typedef}s, etc) should also be 
documented using the docygen tags, but a normalized documentation format has not 
been designed.

\section{Internal documentation}
One should only add comment to internal code (for example to explain a line 
inside the body of a function) after careful consideration. Usually, this 
practice is a sign of bad design (but not always) and poor self documenting 
code. The code should speak for itself. Here are some ideas on how to avoid 
comments in internal code:

\begin{enumerate}

 \item separate the code in smaller functions with meaningful names;
 
 \item code what you want to communicate (for example, use \ttt{const} to say 
       that your method does not modify an object, instead of writting a 
       comment);
       
 \item use well know functions wherever possible (for example, in the excellent 
       \ttt{algorithm} STL header.
       
\end{enumerate}

For example avoid this:
\includecode{\documentationSnippets useless.cpp}

and write this instead:
\includecode{\documentationSnippets inTheCode.cpp}