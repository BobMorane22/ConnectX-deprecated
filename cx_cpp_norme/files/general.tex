% ------------------------------------------------------------------------------
% Connect X project's programming standard general section. 
%
% All general items which do not fit in any other section and that do not (yet 
% at least) diserve a new section should be included here.
%
% @file     general.tex
% @author   Éric Poirier
% @date     March, 2017
% @version  1.0
% ------------------------------------------------------------------------------

% ------------------------------------------------------------------------------
% Copyright (C) 2017 Connect X team.
%
% This file is part of Connect X.
%
% Connect X is free software: you can redistribute it and/or modify
% it under the terms of the GNU General Public License as published by
% the Free Software Foundation, either version 3 of the License, or
% (at your option) any later version.
%
% Connect X is distributed in the hope that it will be useful,
% but WITHOUT ANY WARRANTY; without even the implied warranty of
% MERCHANTABILITY or FITNESS FOR A PARTICULAR PURPOSE.  See the
% GNU General Public License for more details.
%
% You should have received a copy of the GNU General Public License
% along with Connect X.  If not, see <http://www.gnu.org/licenses/>.
% ------------------------------------------------------------------------------

\section{C++11 is the default}
Throughout this document, you will notice that C++11 is used whenever possible. 
This same behaviour is expected from any programmer for the \texttt{Connect X} 
project. C++11 is becoming the new standard in the industry and it is important 
to create as modern a codebase as possible.

For now, C++14 and C++17 features are not to be used (unless there is no 
confortable way around it). When C++17 will become the new standard (that is,
when it will ship with \ttt{gcc}), it will be possible to use it.


\section{Meaningful names}
Try to produce selft-documenting code which requires a minimum analysis from 
the reader. In a perfect world, names given to variables, functions and so on 
should never add to the analysis a user must perform while reading the code. 
Names should be clear and unambiguous. For example:

\includecode{\generalSnippets meaningfulNames.cpp}

Don't fear long names, as long as they are clear.


\section{Source code organization}

The source code should always be organized in multiple files.

\subsection{Classes and structures}
A header file (with the extention \texttt{.h} named after the class it declares 
contains the class declaration but not its implementation. The implementation 
should be in a source file named after the class it implements with a 
\texttt{.cpp} extension. This is also the case for structures.

\subsection{Templates}
Class templates need to have the method declarations inside the header file. To 
keep declaration and implementation separated, do the following:

\includecode{\generalSnippets fileOrganization.cpp}


\subsection{Classes and files}
Keep \textbf{one and only one} class, structure or class template in a given pair of
\ttt{.h/.cpp} files unless they are strongly coupled by concept and their aggregation 
in a unique pair of files simplifies the reader's understanding. Finally, include 
header guards for each header file created. For example, for a header file containing 
the declaration of a class \texttt{Foo}:

\includecode{\generalSnippets headerGuards.cpp}

\section{Language and characters}
To avoid encoding issues, always use English and ASCII characters either in 
your programming or your commenting.

\textbf{IMPORTANT:} tab (\ttt{'\textbackslash t'}) characters should be replaced by 
four (4) space characters. Tab characters may be interpreted differently by other 
applications (such as text editors and IDE) and therefore are a source of ambiguity 
which should be eliminated.