\section{The \ttt{Connect X} Game dialog} \label{dlg:Game}
It is in this dialog that most of the interesting action happens. The following
explain how a user can interract with the \ttt{Connect X} main window:


  \subsection{Menu bar} \label{ssec:MenuBar}
  The menu bar consists of two menus: \ttt{Game} and \ttt{Help}. Both these menus
  are accessible both by clicking on them using the mouse or by using the mnemonics
  \Alt\ttt{+g} and \Alt\ttt{+h} respectively (i.e. pressing both the \Alt and the
  \ttt{g} or \ttt{h} keys simultaneously). Each menu contains the following elements.
  First, the \ttt{Game} menu:

  \begin{description}
    \item[New game]          Brings up the New game dialog (see \cref{dlg:NewGame}).
    \item[Reinitialize game] Reinitializes the current game.
    \item[Quit]              Quits \ttt{Connect X}.
  \end{description}

  \noindent Then, the \ttt{Help} menu:

  \begin{description}
    \item[Show help]         Brings up the Help dialog (see \cref{dlg:Help})
    \item[About Connect X]   Brings up the About \ttt{Connect X} dialog
                             (see  \cref{dlg:AboutCX}).
  \end{description}


  \subsection{Game information area}
  The game information area consists of the part of the dialog (above the game
  board) where information on the game and the current turn is displayed. in this
  section, the following items are displayed\footnote{note that numbers refer
  to the line at which the mentionned item is displayed. Also, items are listed in
  the horizontal order at which they appear on the screen}:

  \begin{enumerate}
    \item[1.] \ttt{Section}, readonly : displays the constant string
                                        'Game information'.
    \item[2.] \ttt{Label},   readonly : displays the constant string
                                        'Active player:'.
    \item[2.] \ttt{Label},   readonly : displays the name of the active
                                        player. This string will change
                                        from turn to turn to adapt to
                                        the new active player.
    \item[2.] \ttt{Label},   readonly : displays the disc color for the
                                        active player.
    \item[3.] \ttt{Label},   readonly : displays the constant string
                                        'Next player:'.
    \item[3.] \ttt{Label},   readonly : displays the name of the next
                                        player. This string will change
                                        from turn to turn to adapt to
                                        the new next player.
    \item[3.] \ttt{Label},   readonly : displays the disc color for the
                                        next player.
    \item[4.] \ttt{Label},   readonly : displays the constant string
                                        'In-a-row value:'.
    \item[4.] \ttt{Label},   readonly : displays the game's in-a-row value
                                        in arabic numerals.
    \item[5.] \ttt{Label},   readonly : displays the constant string
                                        'Number of moves left:'.
    \item[5.] \ttt{Label},   readonly : displays the number of moves left
                                        for the current game in arabic
                                        numerals.
  \end{enumerate}

  \noindent See \cref{fig:Game} to see how these elements should be located in
  the dialog.


  \subsection{Game board area}
  The game board area contains the neccessary tools to play a \ttt{Connect X}
  game. This area can be further divided into subsections (listed in order
  from top to bottom):

  \begin{enumerate}
    \item the section title;
    \item the next disc dropping area;
    \item the game board (sometimes referred as the grid);
    \item the reinitialize button.
  \end{enumerate}


    \subsubsection{The section title}
	The section title is a readonly \ttt{label} that displays the constant string
    'Game board'.


    \subsubsection{The next disc dropping area}
	This section is an horizontal readonly array which is invisible to the
    players and holds either no disc (i.e. before a game is started) or one
    disc: the activeplayer's disc. The array has as many cells as there are
	columns in the game board and sits directly on top (vertically) of the game
    board.


    \subsubsection{The game board}
	The game board is a readonly array for which cells are visible to the players
    and in which discs are stored.


    \subsubsection{The reinitialize button}
	The reinitialize button is a clickable button displaying the constant string
    'reinitialize' before game start, the button is disactivated and unclickable.
    As soon as a game is initialized, it is activated and clickable.


  \subsection{Behaviour} \label{ssec:GameBoardDlgBehaviour}
  The required behaviours for the different elements in this dialog \footnote{The
  behaviour for the menu items has already been described in the
  \cref{ssec:MenuBar} section and have not been repeated here, unless another
  element triggers the same action.} are the following:

  \begin{description}
    \item[Move next disc]  Using the left and right arrow keys (\LArrow and
                           \RArrow), the active player can move the next disc
                           to be dropped (located in the next disc dropping
                           area) to the desired column. This action can also
                           be accomplished using the mouse, by dragging the disc
                           to the disered column, either on the disc dropping
                           area or directly on the gameboard. Note that when a
                           disc is moved, the column it is aligned with is
                           highighted.

    \item[Drop next disc]  Using the down arrow key (\DArrow), the active player
                           can drop a disc from the next disc dropping area to
                           the gameboard. The disc then leaves from the next
                           disc dropping area (it is replaced by the next player's
                           disc) and moves in the gameboard, in column below,
                           just like when a disc is dropped in a Connect 4 game.
                           The players can also use the mouse to drop a disc by
                           releasing the draggued disc in the wanted column. Once
                           the disc has been dropped, the chosen column is no
                           longer highlighted.

    \item[Reinitialize]    To reinitialize the dialog, the user can use the
                           Reinitialize button, or the \ttt{Game->Reinitialize}
                           menu item. Reinitializing a dialog puts the dialog
                           back to the exact same state at which it was set
                           at the game creation (i.e. after a New game dialog
                           was populated with valid parameters and the start
                           button was pressed. See \cref{dlg:NewGame} for more
                           information).

    \item[Close dialog]    To close the dialog, the user can use the standard
                           close dialog button, or use the \ttt{Game->Close}
                           menu item.

    \item[Minimize dialog] To minimize this dialog, the user uses the standard
                           dialog minimize button.

    \item[Maximize dialog] To maximize this dialog, the user uses the standard
                           dialog mmaximize button.
  \end{description}


\section{The new game dialog} \label{dlg:NewGame}
This is the dialog where players can initialize a game. This dialog is of critical
importance to ensure game parameters are set properly. In other words, it is here
that we make sure that class preconditions for the various objects needed in the
backend are respected and that valid objects are created. If those are not met,
the backend behaviour is undefined and all sorts of problems can occur. Refer to
the `cxbase` library documentation for more information. Note that this dialog
should be modal, since game can start before it is completed.


  \subsection{The game area}
  The first dialog area lets players define game parameters. For now, only one game
  parameter exists: the in-a-row value. So the game area is composed of one line
  containing (in horizontal order from left to right):

  \begin{enumerate}
    \item \ttt{Label}   , readonly : displays the constant string 'In-a-row value:'.
    \item \ttt{Edit box}, editable : Takes an integer in the range $[3, 64]$.
  \end{enumerate}


  \subsection{The game board area}
  The second dialog area is the game board area, where players define the game
  board parameters for the game they are about to create. Two parameters define
  a game board: its height and its width. The game board area is then composed
  of two lines, each containing a parameter. They are presented here from top
  to bottom and from left to right:

  \begin{enumerate}
    \item[1.] \ttt{Label}   , readonly : displays the constant string 'Number of
                                         rows:'.
    \item[1.] \ttt{Edit box}, editable : Takes an integer in the range $[6, 64]$.
    \item[2.] \ttt{Label}   , readonly : displays the constant string 'Number of
                                         columns:'.
    \item[2.] \ttt{Edit box}, editable : Takes an integer in the range $[7, 64]$.
  \end{enumerate}


  \subsection{The player registration area} \label{dlg:PlayerReg}
  The third dialog area is meant for players to register themselves. The area is
  composed of a table showing registered players entries and a button to add
  more players.

  The table is an array of dimension 3 by $n + 1$, where $n$ represents
  the number of registered players. There must be a minimum of 2 players
  registered and a maximum of 10. The first row of the array should the
  three readonly column titles: 'No.', 'Name', 'Disc'.

  At dialog start, there should already be two rows in the array in addition to
  the title row, both with a string representing the name of the player
  (to be modified if needed, of course) and a color representing their
  respective disc color. For the two first players, the colors red and black
  should be chosen by default.

  Except for the title row, each row in the registered players table should
  be composed of the following elements (from left to right):

  \begin{enumerate}
    \item \ttt{Text field},     readonly : The player number, an integer in the
                                           interval $[1, 10]$.
    \item \ttt{Text field},     editable : The player name, a text string.
    \item \ttt{Drop down list}, editable : Displays a color (not in text) or
                                           the constant string 'Color...' for
                                           newly added players who have not been
                                           able to choose a color yet. There
                                           should be a pool of at least 10
                                           unambiguous colors to choose from.
                                           Furthermore, each time a player
                                           picks a color, it should no longer
                                           be displayed in the list. If a
                                           player releases a color, it should
                                           re-appear in the list.
  \end{enumerate}

  Each row in the table should be selectable by clicking on the first column
  element for the wanted row. This should highlight the selected row. By hitting
  \Del, a user should be able to unregister a player. The player No. should
  always be updated accordinly. Another way to unregister a player would be to
  right-click anywhere on the row and select 'Remove player' from the menu.

  To register a player, one should use the Add player button. This button should
  be located right below the registered players table and display the constant
  string 'Add player'. This button should be desactivated when a total of 10
  players have been registered. It can be reactivated if players are removed,
  of course.


  \subsection{The start game area}
  The last dialog area is the start game area. From here, players can start the
  game they have just defined, \textit{but only if all parameters in the dialog
  fields are valid}. This is of crucial importance. The button displays the
  constant readonly string 'Start'. It is deactivated until all fields from the
  dialog have been populated.

  In addition to the fied checks already mentionned, this button must perform the
  following checks before starting a new game:

  \begin{enumerate}
    \item The number of players must be smaller or equal to the number of positions
          in the game board divided by the in-a-row value.
    \item The in-a-row value is greater than 2 and smaller than the minimum between
          the number of rows and the number of columns in the gameboard.
    \item The game board dimensions must allow every player to have the possibility
          to play the same amount of moves. In other words, if a draw occurs, all
          players must have placed the same amount of discs.
  \end{enumerate}

  \noindent See the \cref{fig:NewGame} for an example layout. In this example, all
  field conditions are respected.


\section{The help dialog} \label{dlg:Help}
\section{The about \ttt{Connect X} dialog} \label{dlg:AboutCX}
\section{The message dialogs} \label{dlg::message}

%%% Local Variables:
%%% mode: latex
%%% TeX-master: "../connectXGuiSpec"
%%% End:
