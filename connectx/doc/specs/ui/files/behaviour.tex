\section{The \ttt{Connect X} Game dialog} \label{dlg:Game}
It is in this dialog that most of the interesting action happens. The following
explain how a user can interract with the \ttt{Connect X} main window:

  \subsection{Menu bar} \label{ssec:MenuBar}
  The menu bar consists of two menus: \ttt{Game} and \ttt{Help}. Both these menus
  are accessible both by clicking on them using the mouse or by using the mnemonics
  \Alt\ttt{ + g} and \Alt\ttt{ + h} respectively (i.e. pressing both the \Alt
  and the \ttt{g} or \ttt{h} keys simultaneously). Each menu contains the following
  elements. First, the \ttt{Game} menu:

  \begin{description}
    \item[New game]          Brings up the New game dialog (see \cref{dlg:NewGame}).
    \item[Reinitialize game] Reinitializes the current game.
    \item[Quit]              Quits \ttt{Connect X}.
  \end{description}

  \noindent Then, the \ttt{Help} menu:

  \begin{description}
    \item[Show help]         Brings up the Help dialog (see \cref{dlg:Help})
    \item[About Connect X]   Brings up the About \ttt{Connect X} dialog
                             (see  \cref{dlg:AboutCX}).
  \end{description}

  \subsection{Game information area}
  The game information area consists of the part of the dialog (above the game
  board) where information on the game and the current turn is displayed. in this
  section, the following items are displayed\footnote{note that numbers refer
  to the line at which the mentionned item is displayed. Also, items are listed in
  the horizontal order at which they appear on the screen}:

  \begin{enumerate}
    \item[1.] \ttt{Section}, readonly : displays the constant string
                                        'Game information'.
    \item[2.] \ttt{Label},   readonly : displays the constant string
                                        'Active player:'.
    \item[2.] \ttt{Label},   readonly : displays the name of the active
                                        player. This string will change
                                        from turn to turn to adapt to
                                        the new active player.
    \item[2.] \ttt{Label},   readonly : displays the disc color for the
                                        active player.
    \item[3.] \ttt{Label},   readonly : displays the constant string
                                        'Next player:'.
    \item[3.] \ttt{Label},   readonly : displays the name of the next
                                        player. This string will change
                                        from turn to turn to adapt to
                                        the new next player.
    \item[3.] \ttt{Label},   readonly : displays the disc color for the
                                        next player.
    \item[4.] \ttt{Label},   readonly : displays the constant string
                                        'In-a-row value:'.
    \item[4.] \ttt{Label},   readonly : displays the game's in-a-row value
                                        in arabic numerals.
    \item[5.] \ttt{Label},   readonly : displays the constant string
                                        'Number of moves left:'.
    \item[5.] \ttt{Label},   readonly : displays the number of moves left
                                        for the current game in arabic
                                        numerals.
  \end{enumerate}

  \noindent See \cref{fig:Game} to see how these elements should be located in
  the dialog.

  \subsection{Game board area}
  The game board area contains the neccessary tools to play a \ttt{Connect X}
  game. This area can be further divided into subsections (listed in order
  from top to bottom):

  \begin{enumerate}
    \item the section title;
    \item the next disc dropping area;
    \item the game board (sometimes referred as the grid);
    \item the reinitialize button.
  \end{enumerate}

    \subsubsection{The section title}
	The section title is a readonly \ttt{label} that displays the constant string
    'Game board'.

    \subsubsection{The next disc dropping area}
	This section is an horizontal readonly array which is invisible to the
    players and holds either no disc (i.e. before a game is started) or one
    disc: the activeplayer's disc. The array has as many cells as there are
	columns in the game board and sits directly on top (vertically) of the game
    board.

    \subsubsection{The game board}
	The game board is a readonly array for which cells are visible to the players
    and in which discs are stored.

    \subsubsection{The reinitialize button}
	The reinitialize button is a clickable button displaying the constant string
    'reinitialize' before game start, the button is disactivated and unclickable.
    As soon as a game is initialized, it is activated and clickable.

  \subsection{Behaviour} \label{ssec:GameBoardDlgBehaviour}
  The required behaviours for the different elements in this dialog \footnote{The
  behaviour for the menu items has already been described in the
  \cref{ssec:MenuBar} section and have not been repeated here, unless another
  element triggers the same action.} are the following:

  \begin{description}
    \item[Move next disc]  Using the left and right arrow keys (\LArrow and
                           \RArrow), the active player can move the next disc
                           to be dropped (located in the next disc dropping
                           area) to the desired column. This action can also
                           be accomplished using the mouse, by dragging the disc
                           to the disered column, either on the disc dropping
                           area or directly on the gameboard. Note that when a
                           disc is moved, the column it is aligned with is
                           highighted.

    \item[Drop next disc]  Using the down arrow key (\DArrow), the active player
                           can drop a disc from the next disc dropping area to
                           the gameboard. The disc then leaves from the next
                           disc dropping area (it is replaced by the next player's
                           disc) and moves in the gameboard, in column below,
                           just like when a disc is dropped in a Connect 4 game.
                           The players can also use the mouse to drop a disc by
                           releasing the draggued disc in the wanted column. Once
                           the disc has been dropped, the chosen column is no
                           longer highlighted.

    \item[Reinitialize]    To reinitialize the dialog, the user can use the
                           Reinitialize button, or the \ttt{Game->Reinitialize}
                           menu item. Reinitializing a dialog puts the dialog
                           back to the exact same state at which it was set
                           at the game creation (i.e. after a New game dialog
                           was populated with valid parameters and the start
                           button was pressed. See \cref{dlg:NewGame} for more
                           information).

    \item[Close dialog]    To close the dialog, the user can use the standard
                           close dialog button, or use the \ttt{Game->Close}
                           menu item.

    \item[Minimize dialog] To minimize this dialog, the user uses the standard
                           dialog minimize button.

    \item[Maximize dialog] To maximize this dialog, the user uses the standard
                           dialog mmaximize button.
  \end{description}


\section{The new game dialog} \label{dlg:NewGame}
This is the dialog where players can initialize a game. This dialog is of critical
importance to ensure game parameters are set properly. In other words, it is here
that we make sure that class preconditions for the various objects needed in the
backend are respected and that valid objects are created. If those are not met,
the backend behaviour is undefined and all sorts of problems can occur. Refer to
the `cxbase` library documentation for more information.

  \subsection{The game board area}
  The first dialog area is the game board area, where players define the game
  board parameters for the game they are about to create.

  \subsection{The game area}
  The next dialog area lets players define game parameters. 


  \subsection{The player registration area} \label{dlg:PlayerReg}
  The third dialog area is meant for players to register themselves.

  \subsection{The start game area}
  The last dialog area is the start game area. From here, players can start the
  game they have just defined, \textit{but only if all parameters in the dialog
  fields are valid}. This is of crucial importance.

\section{The help dialog} \label{dlg:Help}
\section{The about \ttt{Connect X} dialog} \label{dlg:AboutCX}
\section{The message dialogs} \label{dlg::message}
